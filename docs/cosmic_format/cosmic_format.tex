\documentclass[12pt]{article}
\usepackage[left=2cm,right=2cm,top=3cm,head=4cm,bottom=2cm,a4paper]{geometry}
\usepackage{hyperref}
\usepackage{amsmath}

\begin{document}
\section{Format Description}

For the first processing data product, we choose a fits file format storing the cosmics extracted from a single observation in one binary table.

\noindent Each fits file can thus contain cosmics from up to 511 observations.

\subsection{Binary Table}
The binary table and its columns are:


\textbf{TRACK} A variable length uint32 array of electrons converted to ADU. This encodes the 2D data retrieved from the observation using the \texttt{numpy.flatten} routine.

\textbf{DIM\_AL} Event length in AL

\textbf{DIM\_AC} Event length in AC

\textbf{LOC\_AL} AL Coordinate of Track Element [0,0] on the source image

\textbf{LOC\_AC} AC Coordinate of Track Element [0,0] on the source image

\textbf{TRACK\_EN} Total track energy in electrons

\textbf{DEL\_EN} Uncertainty of total track energy in electrons


\subsection{Header Keywords}

\noindent The keywords of the header, aside from automatically generated ones concerning  are:

\textbf{SOURCE} Source of observation (SM-SIF, BAM-SIF, BAM-OBS)

\textbf{CCD\_ROW} Row of the CCD

\textbf{FOV} FOV (1/2, for all data sources)

\textbf{ACQTIME} Acquisition time [OBMT]

\textbf{SRC\_AL} AL dimension of the source image

\textbf{SRC\_AC} AC dimension of the source image

\textbf{MASKPIX} Number of masked pixels

\textbf{GAIN} CCD gain [e/ADU]


\end{document}
